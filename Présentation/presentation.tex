\documentclass{beamer}

\usetheme{CambridgeUS}
\setbeamertemplate{footline}[frame number] % Affiche uniquement le numéro de page

\setbeameroption{show notes on second screen=right} % Affiche les notes à droite
\usepackage[utf8]{inputenc}
\usepackage{graphicx}
\usepackage{hyperref}
\usepackage[T1]{fontenc}
\usepackage{caption}
\usepackage{placeins} % Required for \Floatbarrier

\title{Edge Computing: A Decentralized Evolution of the Cloud}
\author{Galiléa LE MOULLEC (Mun ID: 202415993) \and Félicien MOQUET (Mun ID: 202415994)}
\institute{Memorial University of Newfoundland, St. John's, Canada}
\date{March 2025}

\begin{document}

\begin{frame}
  \titlepage
\end{frame}

%--- Slide 1: Introduction ---
\begin{frame}{Introduction}
    \begin{block}{\textbf{Definition}}
    Edge computing is a
    distributed computing paradigm that integrates networking, computing, stor-
    age, and application resources \textbf{near data sources} to provide intelligent services
    with minimal delay. By processing and storing data closer to its origin, edge
    computing \textbf{minimizes latency}, \textbf{optimizes bandwidth usage}, and \textbf{enhances system
    responsiveness}
      \end{block}
  \note{
   In this presentation, we will understand Edge Computing. After Cloud Computing has rocked the IT world for the last decade. Now, Edge Computing is picking up pace. With the advent of IoT (Internet of Things) and 5G networks, Edge computing is becoming increasingly important. In this video, we'll understand what Edge Computing is, why it's needed, and how it works, with a real-life examples. So without further ado, let's get started.
  }
\end{frame}

%--- Slide 2: Why Edge Computing? ---
\begin{frame}{Why Edge Computing?}
  \begin{itemize}
    \item \textbf{Latency:} Reduces delay for real-time responses
    \item \textbf{Bandwidth:} Minimizes data transfer volume
    \item \textbf{Privacy:} Keeps sensitive data local
    \item \textbf{Resilience:} Operates even with cloud disconnections
  \end{itemize}
  \note{
    In this video, we'll explore Edge Computing and its importance. Consider a startup that initially relies on cloud computing to manage financial data. As the company grows, real-time decisions are needed, making cloud computing inefficient. Instead, they hire an in-house accountant to process data locally, improving speed and decision-making.

    Edge Computing works similarly, by processing data closer to the source rather than relying on the cloud. Key benefits include:

    - **Latency:** Processing data locally reduces delays, which is vital for applications like smart cars or real-time sensors.
    - **Privacy:** Keeping sensitive data local minimizes the risk of exposure, especially in industries like healthcare.
    - **Bandwidth:** Instead of transmitting all data to the cloud, only relevant information is sent, reducing the need for high bandwidth.
    - **Resilience:** Edge devices continue to function even without cloud connectivity, ensuring uninterrupted service in critical situations.

    These benefits make Edge Computing essential for fast, private, and reliable applications, even when cloud services are unavailable.
  }
\end{frame}


%--- Slide 3: Architecture Overview ---
\begin{frame}{Architecture Overview}
  \begin{itemize}
    \item \textbf{Edge devices:} Sensors, wearables, cameras
    \item \textbf{Edge nodes:} Gateways, micro-servers, local processors
    \item \textbf{Cloud layer:} For large-scale analytics and storage
  \end{itemize}
  \vspace{0.5cm}
  \centering
  \begin{figure}
    \centering
    \includegraphics[width=0.5\linewidth]{IMG/6.png} % Optional
  \end{figure}

  \vspace{0.2cm}
  \small \textit{Figure 1: Edge computing reference architecture}

  \note{
    Let’s now understand the architecture of Edge Computing and how it provides advantages over Cloud Computing.
In Edge Computing, there are three main components: the Cloud Data Center, Edge Gateway Server, and Edge Clients. The smart devices or edge clients are installed in various systems that record and process the data generated by the devices. These devices have built-in intelligence and storage capacity to process the data, and for higher workloads, they send the data to a nearby Gateway Server.
The key difference between sending data to the cloud and sending it to the Gateway Server is that the Gateway Server is physically closer to the device. This proximity reduces latency and increases processing power. Hence, Edge Computing has become popular, especially with IoT applications, connected cars, and smart devices.
The driving force behind Edge Computing is the 5G network because the speed of data processing is increasing, and IoT technology is rapidly advancing. In Edge Computing, the Cloud Data Center, Edge Gateway Server, and Edge Clients work together to process data faster, handle large data volumes, and reduce latency.

  }
\end{frame}

%--- Slide 4: Use Case - IoT ---
\begin{frame}{Use Case: Internet of Things (IoT)}
  \begin{itemize}
    \item Smart homes: temperature, lighting, security
    \item Environmental monitoring: air quality, agriculture
    \item Local processing improves responsiveness and privacy
  \end{itemize}
  \vspace{0.5cm}
  \centering
  \includegraphics[width=0.6\linewidth]{IMG/12.png} % Optional
  \vspace{0.2cm}
  \small \textit{Figure 2: IoT use case}
  \note{
    IoT applications benefit from edge computing through fast reactions and privacy. In smart homes, sensors react immediately. In agriculture, edge devices adapt irrigation in real time.
  }
\end{frame}

%--- Slide 5: Use Case - Autonomous Vehicles ---
\begin{frame}{Use Case: Autonomous Vehicles}
  \begin{itemize}
    \item Onboard sensors generate huge data streams
    \item Requires instant decision-making (e.g. braking)
    \item Edge computing enables safety-critical operations
  \end{itemize}
  \vspace{0.2cm}
  \centering
  \includegraphics[width=0.4\linewidth]{IMG/13.png} % Optional
  \vspace{0.2cm}
  \small \textit{Figure 3: Autonomous vehicles use case}
  \note{
    Autonomous vehicles need to process data within milliseconds. Edge computing allows cars to detect obstacles and make driving decisions instantly, which is essential for safety.
  }
\end{frame}

\begin{frame}{Use Case: Smart Cities}
    \begin{columns}[t]  % [t] pour aligner en haut
      % Colonne gauche : texte
      \begin{column}{0.5\textwidth}
        \begin{itemize}
          \item Real-time traffic management
          \item Public safety and surveillance
          \item Energy optimization and environmental monitoring
        \end{itemize}
      \end{column}
  
      % Colonne droite : image
      \begin{column}{0.5\textwidth}
        \includegraphics[width=\linewidth]{IMG/14.png}
        \vspace{0.2cm}
  
        \small \textit{Figure 4: Smart cities use case}
      \end{column}
    \end{columns}
  
    \note{
      Edge computing enables cities to be smarter and more efficient. Local processing in traffic lights and surveillance systems improves responsiveness and reduces network dependency.
    }
  \end{frame}
  

%--- Slide 7: Use Case - Healthcare ---
\begin{frame}{Use Case: Healthcare and Telemedicine}
  \begin{itemize}
    \item Real-time patient monitoring
    \item On-site diagnostics in emergencies
    \item Strong data privacy and compliance (e.g. GDPR)
  \end{itemize}
  \vspace{0.5cm}
  \centering
  \includegraphics[width=0.6\linewidth]{IMG/15.jpg} % Optional
  \vspace{0.2cm}
  \small \textit{Figure 5: Healthcare use case}
  \note{
    Healthcare systems use edge computing for real-time monitoring and diagnostics. Patient data stays local, enhancing privacy and compliance with health data regulations.
  }
\end{frame}

%--- Slide 8: Pros and Cons ---
\begin{frame}{Advantages and Challenges}
  \textbf{Advantages:}
  \begin{itemize}
    \item Lower latency and bandwidth usage
    \item Better data privacy and security
    \item Improved resilience and scalability
  \end{itemize}
  \vspace{0.3cm}
  \textbf{Challenges:}
  \begin{itemize}
    \item Complex management of distributed nodes
    \item Interoperability with cloud platforms
    \item Security at the edge
  \end{itemize}
  \note{
    While edge computing has many strengths, it introduces new challenges. Managing and securing distributed nodes is complex, and integration with existing cloud systems remains tricky.
  }
\end{frame}

%--- Slide 9: Trends and Future Perspectives ---
\begin{frame}{Trends and Future Perspectives}
  \begin{itemize}
    \item Integration with AI and 5G for smarter edge decisions
    \item Lightweight containers and orchestration (e.g. K3s)
    \item Research in privacy-preserving analytics, federated learning
  \end{itemize}
  \note{
    Edge computing is evolving. With AI, devices make smarter decisions. 5G supports high-speed communication. New tools like K3s make edge deployment easier, and research continues on privacy.
  }
\end{frame}

%--- Slide 10: Conclusion ---
\begin{frame}{Conclusion}
    \begin{itemize}
      \item Edge computing addresses key limitations of centralized cloud
      \item Use cases show strong benefits in latency, privacy, and efficiency
      \item Future: a hybrid cloud-edge ecosystem
    \end{itemize}
    \vspace{0.3cm}
    Thank you!
    \note{
      In conclusion, edge computing complements the cloud. It improves response times, protects data, and enables smarter systems. The future lies in combining both models for flexibility and power.
    }
  \end{frame}

%--- Slide 11: Open Question / Discussion ---
\begin{frame}{Open Discussion}
  \begin{block}{\textbf{Discussion Point}}
    Edge computing reduces data exchanges by processing locally. But:\\
    \textbf{With network demands constantly rising, will edge computing be enough?}\\
    Or is it just a temporary relief before a new saturation point?
  \end{block}
  \note{
    Let's open the floor: Do you think edge computing is a long-term solution, or just a short-term patch? Can it keep up with the exploding demand for connectivity and data?
  }
\end{frame}

%--- Slide 12: References---
\begin{frame}{References}
    \footnotesize
    \begin{itemize}
        \item K. Cao, Y. Liu, G. Meng, et Q. Sun, « An Overview on Edge Computing Research », IEEE Access, vol. 8, p. 85714 85728, 2020, doi: 10.1109/ACCESS.2020.2991734.
        \item \textit{Figure 1}: K. Cao, Y. Liu, G. Meng, et Q. Sun, «An Overview on Edge Computing Research», IEEE Access, vol. 8, p. 85714-85728, 2020, doi: 10.1109/ACCESS.2020.2991734.
        \item \textit{Figure 2}: Nature, "The rise of IoT: Edge computing applications," Sci. Rep. 11, 2021. Available: \url{https://www.nature.com/articles/s41598-021-01431-y}
        \item \textit{Figure 3}: GSA Global, "Edge AI Computing Advancements Driving Autonomous Vehicle Potential," 2023. Available: \url{https://www.gsaglobal.org/forums/edge-ai-computing-advancements-driving-autonomous-vehicle-potential/}
        \item \textit{Figure 4}: StorMagic, "Edge Computing for IoT-Based Energy Management in Smart Cities," 2023. Available: \url{https://stormagic.com/company/blog/edge-computing-for-iot-based-energy-management-in-smart-cities/}
        \item \textit{Figure 5}: CIO Influence, "Best Practices for Integrating Edge Computing in Healthcare," 2023. Available: \url{https://cioinfluence.com/it-and-devops/best-practices-for-integrating-edge-computing-in-healthcare/}
    \end{itemize}
\end{frame}

\end{document}
