\documentclass{article}
\usepackage{graphicx} % Required for inserting images

\title{Script}
\author{}
\date{}

\begin{document}

\maketitle

\section{What is Edge Computing? Why it is in demand? Will edge computing replace cloud computing? (2021)
}


Hello friends. Welcome to IT k Funde, your own channel where we make IT interesting for everyone. In this video, we will understand Edge Computing, after Cloud Computing has rocked the IT world for the last decade. Now, Edge Computing is picking up pace. With the advent of IoT (Internet of Things) and 5G networks, it's becoming increasingly important. In this video, we'll understand what Edge Computing is, why it's needed, and how it works, with a real-life example. So without further ado, let's get started.

Let's take an example. Suppose this is a new startup named ABC, and this guy is the owner of this startup. Now, to manage the financials of this startup, he contacts a consultancy and accountancy firm XYZ. Over the course of the next 4-5 years, this guy manages all the financial data within his company and then sends it over via email or phone. He would also take advice from the consultancy and accountancy firm, and they would maintain all the balance sheets, profit & loss statements, and other documents for the startup.

However, in the span of five years, this startup has grown into a large enterprise with thousands of employees. The business model has become more complex, and the financial information being processed has also become much more complex. Managing the financial health of the company is no longer possible with the remote consultancy. So, the startup talks to the consultancy and asks if they can have a dedicated service all the time because they need to make crucial business decisions daily. They need someone to guide them with the financials.

So, the head of the company decides to deploy a chartered accountant full-time at the firm. This person’s role is to guide the CEO in day-to-day activities, managing financial data, and taking crucial business decisions. It becomes much faster and more efficient. The chartered accountant also sends all the data back to the firm to manage the overall health of the company.

This is a real-life example of Cloud Computing. If you see this as a system, the data processing used to go to the cloud and come back to the system. This was good for a while, but now, as we move into real-time data and IoT applications, processing needs to be faster. This is where Edge Computing comes into the picture.

Let’s now understand the architecture of Edge Computing and how it provides advantages over Cloud Computing.

In Edge Computing, there are three main components: the Cloud Data Center, Edge Gateway Server, and Edge Clients. The smart devices or edge clients are installed in various systems that record and process the data generated by the devices. These devices have built-in intelligence and storage capacity to process the data, and for higher workloads, they send the data to a nearby Gateway Server.

The key difference between sending data to the cloud and sending it to the Gateway Server is that the Gateway Server is physically closer to the device. This proximity reduces latency and increases processing power. Hence, Edge Computing has become popular, especially with IoT applications, connected cars, and smart devices.

The driving force behind Edge Computing is the 5G network because the speed of data processing is increasing, and IoT technology is rapidly advancing. In Edge Computing, the Cloud Data Center, Edge Gateway Server, and Edge Clients work together to process data faster, handle large data volumes, and reduce latency.

We are seeing Edge Computing being implemented in various industries such as agriculture, transport, smart appliances, and self-driving cars. For example, in agriculture, smart devices monitor the health of crops, and if there's an issue like pesticide harm, the devices can immediately stop the spray, providing real-time solutions.

Similarly, connected cars, where smart devices help in making real-time decisions, benefit from Edge Computing to process data rapidly and reduce delays. For smart homes, Edge Computing allows devices like washing machines, refrigerators, and lights to work seamlessly and intelligently.

The major factors driving Edge Computing are the massive amount of data being generated, the advent of 5G technology, and the rise of IoT. These technologies have made it necessary to have faster processing power to make timely decisions.

Edge Computing is a faster, more intelligent way to handle real-time data, offering low latency and better privacy and security. However, there are security concerns as well. While data processing is happening closer to the devices, there's the potential risk of direct hacking into the Edge Gateway Server or the devices themselves.

In conclusion, Edge Computing is a critical technology that will continue to evolve. While it might not replace Cloud Computing, both will coexist and complement each other, enabling faster cloud adoption.

I hope you’ve learned something new about Edge Computing. It's going to be one of the prominent technologies in the coming years. If you found this video helpful, please hit the like button. It helps the video reach a wider audience. And if you want to know more about the world of IT in a simple way, please consider subscribing to my channel "IT k Funde."

Until next time, keep learning, keep sharing knowledge, and as always, keep hustling. Bye for now!
\end{document}
