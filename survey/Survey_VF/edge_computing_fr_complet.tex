
\documentclass[runningheads]{llncs}
\usepackage[T1]{fontenc}
\usepackage{graphicx}
\usepackage{placeins}

\begin{document}

\title{L'informatique en périphérie dans le Cloud Computing}

\author{Galiléa LE MOULLEC (ID Mun : 202415993) \and Félicien MOQUET (ID Mun : 202415994)}
\authorrunning{G. Le Moullec, F. Moquet}
\institute{Université Memorial de Terre-Neuve, St. John's, Canada\\
\email{glemoullec@mun.ca}, \email{fmoquet@mun.ca}}

\maketitle

\begin{abstract}
L'informatique en périphérie améliore considérablement le cloud computing en localisant le traitement des données près des sources, répondant ainsi aux contraintes de bande passante, réduisant la latence et améliorant l'analyse des données en temps réel. Cet article propose une revue détaillée de l'informatique en périphérie, en se concentrant sur ses principes, son architecture, ses défis, ses applications et les tendances futures.

\keywords{Informatique en périphérie \and Cloud computing \and Défis}
\end{abstract}

\section{Introduction}
L’informatique en périphérie (edge computing) a émergé comme une technologie essentielle, venant compléter le cloud computing traditionnel en rapprochant les ressources de calcul des lieux de génération des données. Elle vise à réduire la latence, optimiser la bande passante et améliorer l’efficacité des applications en temps réel, avec un impact majeur dans des secteurs tels que les véhicules autonomes, les soins de santé et les villes intelligentes.

\subsection{Définition}
L’informatique en périphérie est un paradigme informatique distribué conçu pour rapprocher le traitement et le stockage des données de leur source de génération, minimisant ainsi la latence et l’utilisation de la bande passante. Selon l’Edge Computing Consortium (ECC), elle intègre les ressources de réseau, de calcul, de stockage et d’application à proximité des sources de données, fournissant des services intelligents à proximité.

% ... full French version continues ...

\end{document}
